\documentclass{article}

\usepackage[T1]{fontenc}
\usepackage[spanish]{babel}
\usepackage{titling}
\usepackage[margin=1.2in]{geometry}
\usepackage{titlesec}
\usepackage{graphicx}
\usepackage{todonotes}

\titleformat{\section}
	{\bfseries \huge}
	{}
	{0em}
	{}[\titlerule]

\titleformat{\subsection}
	{\bfseries \Large}
	{}
	{0em}
	{}

\titleformat{\subsubsection}
	{\bfseries \large}
	{}
	{0em}
	{}

\titlespacing{\section} %me permite controlar el espaciado de la seccion que le indico
	{0em}
	{3em}
	{1.5em}

\begin{document}
\begin{titlepage}
    \begin{center}
        \vspace*{1cm}
            
        \Huge
        \textbf{Penguin}
            
        \vspace{0.3cm}
        \LARGE
        Reparación y mantenimiento de computadoras   
        
        \vspace{0.5cm}
        \Large
        Un trabajo presentado para la materia de \\
        Sistemas Productivos
            
        \vfill
            
        \includegraphics[width=0.4\textwidth]{linux-logo.png}
            
        \Large
        \textbf{Krapp Ramiro} \\
        Instituto tecnológico San Bonifacio\\
        Departamento de electrónica\\
        \today
        
        \vspace{0.5cm}
        \large {Hecho en \LaTeX}
            
    \end{center}
\end{titlepage}

\setcounter{tocdepth}{3}
\tableofcontents
\noindent\rule{\textwidth}{0.7pt}

%\listoftodos
\pagebreak

\section{Descripción del proyecto}
	El proposito que tiene Penguin para su existencia es el mantenimiento y documentación de equipos informáticos usados en oficinas y escuelas, para que su potencia y capacidad se mantenga competente a nivel tecnológico a lo largo del tiempo.

\section{Justificación del proyecto}
	Hoy en día es fundamental tener una computadora que funcione 
	de forma correcta para desarrollar una gran cantidad de actividades, 
	tanto recreativas, como de estudio y de trabajo.
	
	Sabiendo eso, muchas veces surgen una gran cantidad de inconvenientes,
	como por ejemplo:
	\begin{itemize}
	\item Averio de computadora por mal uso o por malos hábitos.
	\item Desgaste de los equipos, sus componentes y su capacidad de refrigeración por el paso del tiempo.
	\item Desconocimiento ante capacidad de hacer un trabajo de forma más eficiente / más organizada.
	\end{itemize}

	Ante la necesidad de solucionar todos estos inconvenientes, nace Penguin, 
	una empresa especializada en estos casos, con una gran capacidad para 
	manejar distintas situaciones de forma excepcional, lo cual permite 
	a los clientes aprovechar los recursos informáticos de forma eficiente.
	
	En penguin las computadoras son gestionadas mediante una base de datos, 
	permitiendo un control sobre su estado y facilitando en gran medida su mantención.


\section{Objetivos del proyecto}
	\subsection{Objetivo general}
		Mantenimiento y mejora de equipos informáticos, aprovechando 
		los recursos existentes que una organización pueda poseer 
		sin incurrir en gastos innecesarios

	\subsection{Documentacion de los equipos existentes en una base de datos}
		\begin{itemize}
		\item Sus especificaciones técnicas.
		\item El estado actual de los componentes
		\item Los problemas que tenga en el momento de la revisión.
		\item Los tratamientos correspondientes que se apliquen.
		\end{itemize}
		
	\subsection{Mantenimiento de los equipos de refrigeración}
		\begin{itemize}
		\item Cambio de pasta térmica y thermal pads.
		\item Limpieza de coolers y optimización del flujo de aire.
		\end{itemize}
	
	\subsection{Reparación de equipos en mal estado}
		\begin{itemize}
		\item Equipos que no prendan o no den señal.
		\item Equipos cuyo dispositivo de almacenamiento este gastado o defectuoso.
		\item Otros defectos.
		\end{itemize}
	
	\subsection{Instalación y optimización de sistemas operativos}
		\begin{itemize}
		\item Instalación y optimización de sistemas Microsoft Windows.
		\item Instalación y optimización de sistemas GNU/Linux.
		\item Optimización por software
		\end{itemize}
	
	\subsection{Asesoramiento informático}
		\begin{itemize}
		\item ¿Cómo evitar ser infectado por un virus informático?
		\item ¿Cómo mantener los documentos ordenados?
		\item ¿Cómo trabajar más eficientemente?
		\item ¿Cómo solucionar problemas de forma autónoma?
		\item Malas prácticas, cómo dañan a los equipos que usamos y cómo evitarlas.
		\end{itemize}

\pagebreak

\section{Presupuesto}
	El presupuesto final depende de los servicios contratados

	\subsection{Armado de base de datos}
		\begin{itemize}
		\item Armado inicial \dotfill \$37995
		\item Mantenimiento mensual \dotfill \$4999 por més
		\end{itemize}
	
	\subsection{Mantenimiento de la refrigeración}
		\begin{itemize}
		\item 1 equipo \dotfill \$995
		\item 10 equipos \dotfill \$8499
		\item 50 equipos \dotfill \$33590
		\item 100 equipos \dotfill \$59995
		\end{itemize}
	
	\subsection{Reparación de equipos en mal estado}
		\begin{itemize}
		\item 1 equipo \dotfill \$1450
		\item 10 equipos \dotfill \$10499
		\item 50 equipos \dotfill \$44999
		\end{itemize}
	
	\subsection{Instalacion y mantenimiento de S.O.}
		\textbf{Sistemas GNU/Linux}
		\begin{itemize}
		\item 1 equipo \dotfill \$1150
		\item 10 equipos \dotfill \$8499
		\item 50 equipos \dotfill \$34999
		\end{itemize}
		
		\textbf{Sistemas Microsoft Windows}
		\begin{itemize}
		\item 1 equipo \dotfill \$1380
		\item 10 equipos \dotfill \$10199
		\item 50 equipos \dotfill \$41999
		\end{itemize}
	
	\subsection{Asesoramiento informatico}
		\begin{itemize}
		\item 1 persona \dotfill \$2299
		\item 10 persona \dotfill \$19899
		\item 50 persona \dotfill \$87999
		\end{itemize}

\pagebreak

\section{Requisitos del proyecto}
	\subsection{Descripción detallada del proyecto y del producto}
		El proyecto que planea Penguin en una 
		contratación completa del servicio es el siguiente:
		
		\subsubsection{Desarrollar una base de datos}
			Esta base de datos debe ser robusta y confiable que almacene el estado actual de todos los equipos, el lugar donde
			estan fisicamente ubicados, el historial de tickets del equipo,
			los arreglos que se han hecho y sus componentes actuales tales como:
			\begin{itemize}
			\item CPU
				\begin{itemize}
				\item Modelo
				\item frecuencia
				\item cantidad de nucleos
				\item socket
				\end{itemize}
			
			\item Motherboard
				\begin{itemize}
				\item Modelo
				\item socket del CPU
				\item DDR de la RAM
				\item RAM máxima permitida
				\item Frecuencia máxima de la RAM
				\end{itemize}
			
			\item RAM
				\begin{itemize}
				\item Modelo
				\item Capacidad
				\item Frecuencia
				\item DDR
				\item Cantidad de slots ocupados
				\end{itemize}
			
			\item Almacenamiento
				\begin{itemize}
				\item Modelo
				\item ¿SSD o HDD?
				\item En caso de HDD, velocidad de giro de los platos rotantes del disco 
				\item Cantidad de almacenamiento en GB
				\end{itemize}
			
			\item Graficos
				\begin{itemize}
				\item ¿Integrados o dedicados?
				\end{itemize}		
										
				En caso de dedicados
				\begin{itemize}
				\item Modelo
				\item Capacidad de memoria
				\item Frecuencia de la memoria
				\end{itemize}
			\end{itemize}
			
		\subsubsection{Hacer un mantenimiento a la refrigeración de los equipos}
			Consiste en:
			\begin{itemize}
			\item Limpieza del polvo acumulado en el chasis
			\item Limpieza del polvo acumulado en los chasis FAN
			\item Limpieza del polvo acumulado en los CPU FAN
			\item Limpieza del polvo acumulado en los PCB
			\item Cambio de pasta térmica y thermal pads en los circuitos que lo requieran
			\item Desensamblaje de la fuente de alimentación para su limpieza
			\end{itemize}
		
		\subsubsection{Reparar los equipos en mal estado}
			En caso de encontrar un equipo en mal estado, se procede a lo siguiente:
			\begin{itemize}
			\item Diagnóstico del error
			\item Si el error es de hardware, se reemplaza el equipamiento dañado
			\item Si el error es de software, se arregla el error
			\end{itemize}
		
		\subsubsection{Instalar sistemas operativos y realizar optimizacion por Software}
			\textbf{En PCs con Microsoft Windows 10, se procede a}
			\begin{itemize}
			\item Desactivar servicios innecesarios
			\item Desactivar telemetría
			\item Desactivar la instalación autómatica de actualizaciones 
			opcionales que puedan causar inestabilidad en el equipo
			\item Desactivar las animaciones y la reducir la carga gráfica del escritorio
			\item Desinstalar el software basura con el que viene preinstalado el sistema operativo
			\item En caso de tener una instalacion SSD+HDD, mover las carpetas de 
			Descargas, Fotos, Documentos, Musica y video a una particion en el HDD
			\end{itemize}
			
			\textbf{En PCs con GNU/Linux, se procede a}
			\begin{itemize}
			\item Desactivar el login del firewall (por ejemplo, ufw)
			\item Cambiar a un compositor más ligero o desactivarlo para equipos de bajo rendimiento
			\item Bajar el nivel de swappiness a un nivel más adecuado
			\item En caso de que sea necesario, cambiar a un desktop enviroment más ligero
			para reducir la carga sobre el CPU y la GPU
			\item En caso de tener una instalacion SSD+HDD, montar la particion /home sobre el HDD
			\item Dependiendo de la cantidad de RAM, se pueden considerar los siguientes aspectos:
				\begin{itemize}
				\item En caso de que la RAM sea mayor a 8GB: Montar /tmp sobre una partición tmpfs
				\item En caso de que la RAM sea mayor a 6GB: Reducir el inode cache de forma menos agresiva modificando sysctl.conf
				\item En caso de que la RAM sea menor a 4GB: Incrementar zswap
				\end{itemize}
			\end{itemize}
		
		\subsubsection{Asesorar informaticamente sobre los siguientes temas}
			\begin{itemize}
			\item Virus informáticos
				\begin{itemize}
				\item ¿Qué es un virus informático?
				\item ¿Cómo se transmite?
				\item ¿Cómo se evitan?
				\item Como compartir y recibir documentos evitando infecciones
				\end{itemize}
			
			\item Organización de documentos
				\begin{itemize}
				\item ¿Cómo asignar nombres de forma correcta?
				\item ¿Cómo ordenar las carpetas de manera idonea?
				\item ¿Cómo mantener un historial de las versiones de un documento?
				\item ¿Cómo mantener un escritorio prolijo?
				\end{itemize}
			
			\item Trabajo de forma eficiente
				\begin{itemize}
				%\item ¿Cómo regular el tiempo de trabajo y de recreación?
				%\item ¿Cómo crear un espacio de trabajo cómo y comfortable?
				%\item ¿Cómo mantener el espacio de trabajo ordenado?
				%\item ¿Cómo mantener una buena postura?
				\item ¿Cómo hacer documentos y presentaciones de forma eficiente con {\LaTeX}?
				\item Los atajos de teclado y cómo ayudan a trabajar de forma rápida
				\end{itemize}
			
			\item Solución de problemas de forma autónoma
			 	\begin{itemize}
			 	\item ¿Cómo buscar cosas en internet filtrando resultados?
			 	\item Codigos de error y cómo buscarlos en internet
			 	\end{itemize}		
			
			\item Malas prácticas
				\begin{itemize}
				\item Cómo mantener una buena ventilación en los equipos
				\item Como evitar dañar los equipos
				\end{itemize}	
			\end{itemize}			
		\newpage	
		
	\subsection{Alcance del proyecto}
		El proyecto en general no incluye
		\begin{itemize}
		\item La instalación eléctrica, de gas y de agua
		\end{itemize}
		
		\subsubsection{Mantenimiento de los equipos de refrigeración}
			Incluye:
			\begin{itemize}
			\item Limpieza de chasis FAN y de CPU FAN
			\item Limpieza de fuente de alimentación
			\item Cambio de pasta térmica y thermal pads
			\item Limpieza superficial de PCB's
			\item Limpieza superficial del chasis
			\item Limpieza de las pantallas
			\end{itemize}
			
			No incluye:
			\begin{itemize}
			\item Limpieza del lugar de trabajo
			\end{itemize}
			
		\subsubsection{Reparación de los equipos}
			Incluye:
			\begin{itemize}
			\item Reparación por desperfecto técnico de fabricación
			\item Reparación por el mal uso de los equipos
			\end{itemize}
		
		\subsubsection{Instalación y optimización de los sistemas operativos}
			Incluye:
			\begin{itemize}
			\item Instalación de Windows 10 LSTC
			\item Debloatting de sistemas Microsoft
			\item Instalación de sistemas GNU/Linux basados en Debian, Red Hat o Arch
			\item Optimización de sistemas GNU/Linux basados en Debian, Red hat o Arch
			\end{itemize}
			
			No incluye:
			\begin{itemize}
			\item Instalación y optimización de los siguientes sistemas:
				\begin{itemize}
				\item Microsoft Windows XP / Vista / 7 / 8 / 8.1
				\item GNU/Hurd
				\item Sistemas GNU/Linux sin systemd como Artix
				\item Sistemas GNU/Linux con graficos integrados SiS
				\item Sistemas GNU/Linux no basados en Debian, Red Hat o Arch como:
					\begin{itemize}
					\item Slackware
					\item NixOS
					\item Void
					\item Alpine
					\item Gentoo
					\item OpenSUSE
					\item Deepin
					\end{itemize}
				\end{itemize}
			\end{itemize}
		
		\subsubsection{Asesoramiento informático}
			Incluye:
			\begin{itemize}
			\item Capacitación sobre la protección contra software 
			malicioso, páginas web maliciosas y como evitar caer en pishing
			\item Capacitación sobre los buenos hábitos a la hora
			de mantener documentos informáticos bien ordenados, 
			organizados y categorizados
			\item Capacitación sobre backups y precaución
			ante la posible pérdida de datos
			\item Capacitación sobre cómo trabajar de forma
			más eficiente y profesional
			\item Capacitación sobre como solucionar los problemas
			de forma autónoma y sin la necesidad de requerir a un técnico
			\item Capacitación sobre como mantener un equipo para evitar
			desperfectos y lograr que mantenga un correcto funcionamiento
			a lo largo del tiempo
			\end{itemize}
			
			No incluye:
			\begin{itemize}
			\item Capacitación sobre lenguajes de programación
			\item Cuestiones ajenas a la informática
			\end{itemize}
	\subsection{Objetivos del proyecto}
		\subsubsection{¿Qué tareas debemos realizar?}
			Se debe, segun lo contratado:
			\begin{itemize}
			\item Hacer una base de datos
			\item Mantener los equipos de refrigeracion
			\item Reparar los equipos en mal estado
			\item Instalar y optimizar los sistemas operativos
			\item Asesorar informaticamente a los empleados
			\end{itemize}
			
		\subsubsection{¿Cuándo se realizarán las tareas?}
			Las tareas se comenzarán a realizar pasada una semana desde la firma del contrato 
			como máximo, y su finalización dependera de los servicios contratados.
			
		\subsubsection{¿Qué recursos son necesarios para realizarla?}
			\textbf{Para la base de datos se necesita}
			\begin{itemize}
			\item Una computadora capaz de manejar el tamaño de la base de datos\\
			\end{itemize}
			
			\textbf{Tanto para la reparación como para el mantenimiento se necesita, por persona:}
			\begin{itemize}
			\item 1 juego de destornilladores
			\item 1 pinza y alicate
			\item 1 aplicación de pasta térmica
			\item Thermal pads de multiples tamaños y grosores
			\item 1 cepillo antiestática
			\item Multiples precintos
			\end{itemize}
			En caso de que haya componentes averiados o malfuncionantes, su correspondiente reemplazo\\
			
			\textbf{Para la instalación y optimización, se necesita}
			\begin{itemize}
			\item 1 pendrive booteable (preferiblemente con Ventoy y multiples ISO's) 3.0 de 64GB
			\item 1 juego de mouse, teclado y parlantes 
			\item 1 pantalla
			\item Scripts necesarios para las optimizaciones
			\end{itemize}
			
		\subsubsection{¿Quién es responsable del cumplimiento de la tarea?}
			\begin{itemize}
			\item Reparación, instalación y mantenimiento --- Los técnicos de Penguin
			\item Documentación y base de datos --- El equipo de Base de datos de Penguin
			\item Asesoramiento informático --- El equipo dedicado a asesoramiento de Penguin
			\end{itemize}
			
		\subsubsection{¿Cuándo deben estar completas?}
			Depende del plazo estipulado en el contrato y del tamaño de la empresa
			
		\subsubsection{¿Qué resultado se espera de la tarea?}
			Se espera un sencillo resultado, tener un equipo informático robusto y potente, bien
			organizado y con los estándares de hoy en día
			
\section{Descripción del alcance del producto o servicio}

\section{Entregables del proyecto}

\section{Restricciones del proyecto}


\end{document}